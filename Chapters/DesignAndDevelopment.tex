% Chapter Template

\chapter{Design and Development} % Main chapter title

\label{DesignAndDevelopment} % Change X to a consecutive number; for referencing this chapter elsewhere, use \ref{ChapterX}

\lhead{Chapter \ref{DesignAndDevelopment}. \emph{Design and Development}} % Change X to a consecutive number; this is for the header on each page - perhaps a shortened title

%----------------------------------------------------------------------------------------
%	SECTION 1
%----------------------------------------------------------------------------------------

Describe the process of development,
and which parts of \theartefact\ which were developed during the different stages of development.
I am then going to give a high level overview of the system.
After giving an outline of the development process and the system,
the main parts of the system will be described in greater detail.

\section{The different phases of development}
This section will outline the main sequence in which work on \theartefact\ was done.
The overview of the iterations will explain when the initial development of
the different part were to give an short explanation of the development process.

\subsection{Iteration 1}
The first iteration started with a technical peak exploring different tools and frameworks for development with JavaScript.
I wanted to find tools that could reduce development time by reducing the amount of repetitive work done in the project.

It was decided that the Web page should use Twitter bootstrap\footnote{\url{http://twitter.github.io/bootstrap/}} as a grid system.
Using a standard grid system makes it easier to create a Web page that looks OK,
and reduces the amount of time needed to set up a scaffold.
Since the visual aspect of the Web page was not an important aspect of the work a simple grid system like bootstrap
eased the development burden.

It was decided to use SASS\footnote{\url{http://sass-lang.com}} to write the style sheets.
It has a syntax that is similar to that of CSS.
In addition to the regular syntax of CSS it also includes the possibility of using variable.
This makes it possible to make global changes while only modifying one line in the style sheet.
SASS compiles to standard CSS.

JavaScript lacks a good IDE so it would be helpful to have some tool that would check the code to make sure that the
code was consistent and that the syntax was correct.
I decided to use JSLint\footnote{\url{http://www.JSLint.com}} to validate the code.
Using a linter eases development as it enforces a particular coding standard for the project.
The way the project was set up, the project wouldn't build without it passing the linting test.

Since the project depended on compiling the style sheets and linting the code it was decided to use grunt.js\footnote{\url{http://gruntjs.com}}
to automate this process.
Grunt.js is a task manager that can run certain tasks when specified events are triggered.
The project was set up so that each time a JavaScript or SASS file was saved the task manager compiled the SASS document,
and concatenated the public JavaScript files.
Concatenating the JavaScript files meant that it was unnecessary to update the HTML when a new JavaScript file was created or deleted,
since the Web page only needed to include that one file.

Node.js was chosen as the development platform of the project.
Setting up a basic server in node.js can be done in a single line of code.
Using this basic server requires a lot of low level handling of requests and responses,
including parsing request URL to find the correct handler.
To abstract these low level concepts away \theartefact\ uses express\footnote{\url{http://expressjs.com}},
a Web app framework which simplifies routing requests to the correct handler and handling static files.
It was also decided to use the jade\footnote{\url{http://jade-lang.com}}
templating language to generate the HTML that was used on the Web site.
Using a templating language made the distinction between structure and content clearer,
and enabled reuse of structure and content between different versions of the Web site during development.

The technological peak also included getting to know the DOM, the API used for manipulating HTML documents.
The problem that was tried to solve during this technical peak was how to find minimal legal ranges.
\Theartefact\ has to create separate tags to add the metadata it created,
and that the system was going to let users select which portions of the text that carried the semantic content.
Since the user selection might not be a legal HTML range I worked on an algorithm to find the smallest range that
was both a legal range, and which contained the whole of the selected text.


\subsection{Iteration 2}
The next step consisted of actually generating metadata tags with the correct RDFa syntax that also enclosed the user selection.
One of the tasks was giving each selected section a unique id so that they could be referenced externally.
It was decided that this id should carry some information about the content of the tag where possible,
to increase the readability for the users.
The mapping files had not been created at this point so the tags contained mock data.

In this iteration work also began with generating the hypernym chains that were needed to find mappings.
The initial approach was to use an existing triple store and query the information with SPARQL.
The triple store used an early version of SPARQL that did not allow recursive queries.
Since the system needed to find the transitive closure of the hypernym relation,
using this source would require multiple and sequential queries and would take to long to be tenable.
JavaScript does not have a mature library for querying the WordNet database,
so it was not possible to not create JavaScript module to perform the task.
The solution used was instead to use Perl to write a script that could query a local WordNet instance.
This again required a technical peak to learn the basics of the Perl programming language but gave a satisfying solution.

This iteration was also used to find WordNet mapping files and convert them to a format more suitable for the system.
I had a mapping file between WordNet and SUMO in the WordNet database format,
and one between WordNet and schema.org in RDF.
Both of these files were translated into JavaScript objects.

\subsection{Iteration 3}
In the third iteration the actual best-fit algorithms that my supervisor and I had proposed were implement.
The algorithms will be discussed further in section \ref{BestFitMapping}.
The two strategies used by these algorithms were to either look at all the hypernyms of the synsets before looking at other senses,
or to look alternately at the direct hypernym and then the siblings of the synset.

This was also the iteration where development started on  a mechanism for fetching HTML from other Web sites so that they could be
marked up using \theartefact.

\subsection{Iteration 4}
The export module was developed in the fourth iteration.
This module would take the Web page that the user had imported into the system and create a new HTML document.
This document would be stored in the database, and would be accessible for the user through a URL which was provided when the page was exported.

Adding properties to entities was also developed during this stage.
This was only implemented for schema.org properties,
both to keep the number of properties at a manageable level,
and because extracting all the allowed properties for all SUMO classes would be a difficult task.

In this iteration the ability to add author information was also added.
This is not closely tied to the problem of translating natural language to formal ontologies,
but it is tied to allowing users to add metadata in a simple way without knowing about the formal underpinnings.

\subsection{Iteration 5}
The last iteration consisted of testing and refactoring the system.
I wrote some test scripts that would run the best-fit algorithms that had beed created over a set of synsets,
and print out the resulting mappings along with information about whether the algorithms gave corresponding results,
and the measure of quality of the mapping.
The results of these tests will be discussed further in section \ref{ComparingAlgorithms}.

\subsection{Overview completed system}
The artefact was divided into several modules which function independently of each other.
The modules were loosely coupled so that each module could be changed or modified without affecting the other modules.
The modules that were created were:
\begin{itemize}
	\item The Web front end
	\item A module that fetched and processed the HTML from the sites that the user wanted to mark up
	\item A module for fetching possible disambiguating terms from lexitags
	\item A module for finding the best-fit mappings for SUMO and schema.org
	\item A module for finding the best-fit mappings from DBPedia to schema.org
	\item A module that puts the document back in its initial state and saves it to a persistent storage
\end{itemize}

\section{Interaction method}
\label{Interaction}
When first faced with the problem of how to let users interact with the Web app I thought that using the highlighting or
selection of text to be the most intuitive approach.
This also appeared to be the simplest interaction method to implement.
The simple case of taking selected text and displaying it in the console can be implemented as in listing \ref{DOMSelection}.

\begin{lstlisting}[caption={Logging selected text}, label=DOMSelection]
	var logSelection = function () {
		console.log(window.getSelection().getRangeAt(0).toString());
	};
	document.addEventListener('mouseup', logSelection);
\end{lstlisting}

The "mouseup" event is triggered every time the user has pressed and then released the mouse pointer on the Web site.
This event is not available on mobile and tablet type browsers,
but adding listeners for corresponding events should not be difficult if there should be a need for this at a later stage.

In the Web app the selected text is sent to the Web server to find possible disambiguations for the text.
The lexitags service that the system uses for disambiguation now is targeted towards disambiguating single words.
It does handle some composite terms and some proper nouns, but this is outside the normal use case.
The lexitags service is designed to disambiguate single tags, not large paragraphs of text.
To accommodate this \theartefact\ does a simple check for the length of the text string.
If the length of the string is more than the longest allowed length,
the query is instead replaced with a shorter query indicating that the selection is of a larger section of text.

The response from the server is a JSON string containing the terms that were found to be possible senses of the selected text.
Each of these senses have an explanation property which gives a short textual description of the meaning of the sense.
These senses are displayed as a list on the Web site.
The senses that are returned come from different sources.
They can come either from WordNet and take the form of WordNet synsets, or they can be resources from DBPedia.
Both of these can be used to generate mappings,
but the most interesting mapping for this thesis lies in generating mappings from WordNet so the synsets are given preference in the list.
DBPedia does not have a standardized way of finding the super type of a resource,
so if the resource does not have a schema.org mapping it is difficult to find generalizations that do.
Lexitags also returns the top-level resources from schema.org as shown in figure \ref{TopLevelSchemaOrg}(page \pageref{TopLevelSchemaOrg}).
These are sorted to the bottom of the list as these are fallbacks for when none of the of the terms suggested for
the word turn out to be reasonable suggestions for the selected text.

A special case for images was also developed,
as these are object on a Web page that were assumed users would be interested in marking up,
but do not contain any text that can be disambiguated.
This special case kicks in when the selection highlighted contains an image tag, but no text.

\subsection{Adding types}

To select one of the terms from the list as the meaning of the selected text the user clicks the sense in the list see in
figure \ref{SidebarMeanings}.
\fig{SidebarMeanings}{The list of possible interpretations of "discoveries"} % IMAGE OF LIST OF INTERPRETATIONS
When a sense is clicked it is sent to the server to be mapped.
The servers uses the best-fit algorithm described in \ref{BestFitMapping}(page \pageref{BestFitMapping}) to find the ontology references
that fits the sense best and attaches these references to the selection using RDFa.
The algorithm used to attach the metadata to the selection can handle an arbitrary amount of namespaces and references.
If there is duplication of references these will be dropped.

When the metadata element is created,
the schema.org type is sent back to the server to get a list of the possible properties of that type.
The server uses a JSON representation of the schema.org ontology to find the properties that a type can have.
The representation that is used on the server was retrieved from \url{http://schema.rdfs.org},
which is a support site that tries to promote the use of linked data.
For each of the types in schema.org the JSON file provides the properties specific to that type,
and all the properties it has inherited from its super types.
The information about which properties a type can have is combined with a short description of what the property represents,
and information about the range of the property, that is the range of values that are valid values of the property.
The resulting JSON object is then returned to the Web site.
These properties are then added to the metadata element as separate elements,
with the description and range stored as data attributes of the element.

\subsection{Adding properties}

Users can click text that has been tagged with metadata.
This triggers a popover view that displays the properties that the element can have,
and which types of values are allowed for each property as shown in figure \ref{PropertiesFull}.
\fig{PropertiesFull}{{The properties popover for schema:Event}} % IMAGE OF PROPERTY SELECTION BOX
If the property allows other schema.org types as their value the Web application does a scan of the content of the Web site to check if
there are elements of the correct type on the page.
Elements of the correct type that are found are put in a combo box and can be selected as values for the property.
In all cases the user is presented with the option of writing the value of the property in an input field.
If the property already has a value it will be displayed in the input field when the popover is displayed.

%\fig{Sidebar}{The information tab of the sidebar},

\section{Maintaining the well-formedness of the document}
One of the difficult aspects of adding the metadata to the selections was making sure that the HTML was well-formed
after the metadata was inserted.

Special care needs to be taken with the ranges of the selection.
Since the system need to insert the metadata into tags it can't just use the user range without some checks.
If we consider the simple case:

\texttt{<p>some text \hl{in a p</p> <p>` plus some} other text </p>}

If the Web application simply inserted a tag it would produce invalid HTML since it would create overlapping elements.

\texttt{<p>`some text <tag> in a p</p> <p> plus some </tag>  other text </p>}

One way to get a well-formed HTML document would be to close and open the HTML tags.

\texttt{<p>some text </p><tag><p> in a p</p> <p> plus some </p></tag><p>  other text </p>}

Solving the problem in this manner would change the structure and rendering of the Web page,
as it would create new block elements.
This is not a tenable solution, as one of the goals of of this thesis is to not change the way the document is displayed.
The solution that was found was to instead expand the range of the selection so that it covered a larger part of the document:

\texttt{<tag><p>some text in a p</p> <p> plus some other text </p></tag>}

The interaction method consists of letting the user make arbitrary selections and then adding metadata to the selection.
This could easily lead to overlapping elements in the HTML as shown above.
To avoid this malformed HTML the system checks the range of the selected text and see if surrounding it with a tag
containing metadata would result in a well-formed document, and modify the range if needed.
If the start and end of the selection are in the same element then adding metadata is safe.
If that is not the case then the Web application will find the smallest change that can be made to the scope of the range to make it safe.
There are two general cases her, when either the start or the end of the range is in a descendant element of the other,
and when both start and end are descendants of a common element but not of each other.

In the first case the system needs to find out which element is the descendant of which.
This is done by following the ancestor links of each element.
When the artefact finds that either the start or end element has the other as an ancestor it uses the ancestor which is the direct
child of the other element and place the start or end tag right before or after that tag, depending on which element
descended from which.

When the elements are not descendants of each other then the only option is that they must be descendants of some other element.
If nothing else they must both be descendants of the root element.
The approach for this is similar to what is done for elements that are descendants of each other,
but instead of finding the ancestor which is the direct child of the other node the Web application finds the one which is a child of the
closest common node.
These elements are then surround with the metadata tag.
This metadata element is styled with a distinct color to let the user know that the text is tagged.

\section{Importing Web pages}
One of the issues that came up early was how to import and display the Web sites that the user wanted to add metadata to.
Early attempts tried using the iframe\footnote{http://www.w3.org/TR/html5/embedded-content-0.html\#the-iframe-element}
element of \nom{HTML}{HyperText Markup Language}.
Embedding the target Web site in an iframe was the first try as it would allow the page to display in the same manner
as it does when accessed directly.
Using the iframe element in this manner would however constitute cross-domain communication and is not allowed.
The solution that is used in the system now uses a module on the server to fetch the HTML of the document the user wants to enrich.
The HTML that is imported is now appended to a content element in the Web application.
The JavaScript events used in the Web application are attached to this element,
to keep the interaction confined to the content the user is meant to tag.
The id of this tag is also used to tell if a certain element is a part of the user content section by checking if it is
the child of the element.

The user writes the URL of the Web page into an input field on the top of the Web application as shown in figure \ref{MadameHeader}.
\fig{MadameHeader}{The header of the \theartefact\ Web application} % IMAGE OF THE HEADER OF MADAME
The server will fetch the document stored at that location.
I created a proxy to get the html from other Web sites.
The proxy would take a URL and get the resource located there.
If the resource is not of the right type i.e. not a HTML document with a body element,
the proxy would return an error.
When the resource has been downloaded the content of the body element of the document is parsed to comment out code that
could be harmful, that could make the page display incorrectly, or that could disrupt the way the Web application functions.
The parts of the page that are comment out are the script, iframe and comment elements.
The script elements could disrupt the way the Web app works, either by overwriting the behavior of functions,
or otherwise alter the behavior of the Web app.
In addition the script elements are not visible on the page so interacting with them would be hard.
Removing them was also convenient during development.
The scripts in the tags would often rely on variables or functions that should have been loaded elsewhere,
and they would clutter the console making it more difficult to find the relevant information.
Iframes were excluded since they could make the Web app appear to be inconsistent.
The iframes look like a part of the page,
but it would not be possible to alter their content since the HTML belongs to a different document.
The difference between the document and the iframe would be invisible to the user,
so it was decided that it would be better to remove the element all together.
The comment elements were removed to make sure that the page would display the way it should.
Since the system uses comments to remove iframes and scripts, and since these elements could them selves contain comments,
the content of the comments would sometimes be displayed as text elements on the page.

\Theartefact\ is only designed to add metadata to the body element of the document,
so the page is split into body and head parts before it is returned to the Web application.
The document is returned as a JSON string with one property containing the processed body,
and one containing the content of the head.


\section{Generating mapping files}
An important part of the early work was to generate files that the tool could use to find mappings from WordNet synsets
into the ontologies which were used.
The ontologies that were chosen for the project, SUMO  and schema.org, already had these kinds of mapping files.
The mapping files were however in different formats and in formats that were unsuitable for use in a JavaScript based application.
The format that was chosen was one that provided the option to ask if there was a mapping for a given WordNet synset,
in other words what was chosen was a dictionary.
JavaScript objects are themselves dictionaries, making it natural to choose to them as the representation.
The structure of the objects is very simple,
the synsets are used as the keys to entities in the target ontologies as seen in listing \ref{wn2schemaMapping}.

\begin{lstlisting}[label=wn2schemaMapping, caption={Excerpt from the WordNet to schema.org (\href{https://github.com/EivindEE/Madame/blob/master/mappings/wn2schema.js}{wn2schema.js}) mapping file}]
exports.mapping = {
	"abstraction#n#6": "Intangible",
	"accounting_firm#n#1": "AccountingService",
	"address#n#6": "PostalAddress",
	// The rest of the mappings
	"work_unit#n#1": "Energy"
};
\end{lstlisting}

Since \theartefact uses JavaScript objects to represent the mappings,
querying a mapping object to check if it contains a mapping for a certain synset would then just consist of checking if
the object has a property with the given name as in listing \ref{LabelExists}.
\begin{lstlisting}[label=LabelExists,caption=Testing if a mapping exists]
	if (mapping['synset-to-check']) {
		// it contains a mapping
	} else {
		// it does not contain a mapping
	}
\end{lstlisting}

The mapping files were to large to translate to JavaScript objects by hand, with the SUMO mapping file alone
covering more than 82.000 mappings, so I used regular expressions to modify the files globally.
The schema.org mapping file\footnote{\url{https://github.com/mhausenblas/schema-org-rdf}}
that was used as the basis for the wn2schema mapping file is written in \nom{RDF/XML}{An XML representation of RDF}.
The first step in creating the mapping file was to remove the parts of the file that were irrelevant to the problem at hand.
The original file contains triples describing schema.org properties,
the schema.org entities and the hierarchal ordering of entities.
This information is handled at other places in \theartefact\ and could be safely removed.
Only the elements mapping synsets to schema.org terms were kept, and all other information except the names were removed.
At this point the XML tags were stripped so only the synset reference and the schema type remained.
Since the correct elements were now at the correct place it was only necessary to surround them with the correct quotation
marks, separate them with a colon and add a comma at the end to separate it from the next property.

The SUMO\footnote{\url{http://sigmakee.cvs.sourceforge.net/sigmakee/KBs/WordNetMappings/}} mapping followed a similar pattern,
but the original format was different as it used the WordNet database format.
The WordNetMappings30-noun file was used as the basis so for this mapping so removing non-noun phrases was unnecessary.
The fact that each line in the file corresponded to a line in the mapping file made the process of translation easier,
there was however one caveat in that the file used the offset to identify the synset,
instead of the word\#category\#number format which is used for the other mapping file.
I put some work into finding a way of substituting the offset for the synset id,
but found that it was simpler to find the offset for the synsets that were queried.
Unifying the way the mapping object properties are named is one of the things that can simplify
the generalization of the process at a later stage.



\section{Building the hypernym chain}
\label{BuildingTheHypernymChain}
Many synsets that the users will select as representing the concepts in the Web app will be more specific than
the ones used in ontologies \theartefact\ maps to.
This means that the tool needs some way of generalizing the synsets.
This thesis examines two approaches to finding mappings from the synset to the ontologies.
One approach is to use the hypernym chain,
a term used in \citet{Veres2011} to describe the list of all the hypernyms of a synset.
The other approach examined is to use the sibling senses of the synset and the synsets in the hypernym chain.

JavaScript does not have a library for 	querying the WordNet database,
so this part of the project needed to be coded in another language.
Had there been a JavaScript library for querying WordNet,
it would be natural to generate the hypernym chain while searching for the mappings from the synset to the ontologies.
Having to use a different programming language and calling it outside the program,
it is easier and more efficient to split the process.
The system will first generate the entire hypernym chain and add the siblings of all of the hypernyms,
and then examine the hypernym chain to find the best mappings.
This section will describe the first part of this process, generating the hypernym chain.

The system uses a Perl script\footnote{\url{https://github.com/EivindEE/Madame/blob/master/scripts/parents.pl}} to generate the hypernym chains.
The script is called by using the node
child\_process.exec\footnote{\url{http://nodejs.org/api/child\_process.html\#child\_process\_child\_process\_exec\_command\_options\_callback}}
function, which calls a command line utility and uses the output stream as the input to the callback function.

The structure of WordNet as a whole can be described as a directed graph, as shown in figure \ref{WordNetGraph}.
The graph must also be acyclical since the hypernym relation is asymmetric as mentioned in section \ref{Hypernymy}.
\fig{WordNetGraph}{Part of the WordNet graph}
Each synset can be described as a node in this graph, with a directed edge connected to its direct hypernym.
For each node in the graph it is also true that following one of the directed edges will lead to the synset {entity},
which is the most general concept in WordNet.

The hypernym chain the script generates is a list of objects.
The objects contain the hypernym itself, its offset in WordNet, and a list of all its siblings.
Each sibling is stored as an object containing the synset, and its offset.
An example of how the hypernym chain is represented as JSON can be seen in listing \ref{PerlHypernymChain}.

\begin{lstlisting}[float=t, label=PerlHypernymChain, caption={Excerpt from the hypernym chain for student\#n\#1}]
{
	"chain": [
		{
			"synset": "enrollee#n#1",
			"siblings": [ { "synset" : "self#n#2", "offset" : "09604981" ...},
			],
			"offset": "10059162"
		},
		{
			"synset": "person#n#1",
			"siblings": [],
			"offset": "00007846"
		}
	],
	"synset": "student#n#1",
	"siblings": [],
	"offset": "10665698"
}
\end{lstlisting}

Generating the hypernym chain is done in two steps.
The first step is to create a list of all the hypernyms,
while the second is to add all the siblings.
To generate the list of hypernyms, the script does a breadth first traversal of the graph as described in listing \ref{HypernymChainCode}.
For the graph in figure \ref{WordNetGraph} the algorithm would add \{living thing\},
and \{causal agent\} before adding \{physical entity\} and \{entity\} to the list.
If a synset is encountered multiple times, it is ignored in subsequent encounters.
The hypernyms are added to the end of the list when they are found.
This means that the synsets that are at the start of the list,
are the ones that are at the closest level of abstraction to the synset for which the chain is built.

\begin{lstlisting}[float=t, language=perl, label=HypernymChainCode, caption={Excerpt from the hypernym chain Perl script}]
my @hypernym = hypernym($synset);
my @hypernyms;
while ( @hypernym){
	my $hypernym = shift @hypernym;
	push(@hypernym, hypernym($hypernym));
	push(@hypernyms, $hypernym);
}

my @hypernyms_with_siblings;
push(@hypernyms_with_siblings, {"synset" => $_ , "offset" => $wn->offset($_) ,"siblings" => [siblings($_)]}) for @hypernyms;

my %result_map = ("chain" => [uniq(@hypernyms_with_siblings)], "synset" => $synset, "offset"=> $wn->offset($synset), "siblings" => [siblings($synset)]);
my $json = JSON->new->utf8->pretty->encode(\%result_map);
print $json;

sub hypernym {
	my @hypernym = $wn->querySense($_[0], "hype");
	if ($#hypernym > 0) {
		return ();
	}
	return @hypernym;
}
sub siblings {
	my $synset = $_[0];
	my @hypernym = hypernym($synset); # Find hypernym of synset
	my @siblings;
	my @sibling_and_offset;
	push(@siblings, $wn->querySense($_, "hypo")) for @hypernym;
	for my $sibling (@siblings) {
		push (@sibling_and_offset, {"synset" => $sibling, "offset" => $wn->offset($sibling)});
	}
	return @sibling_and_offset;
}
\end{lstlisting}


After the list of hypernyms has been created it is iterated over to add the siblings of each of the hypernyms.
For each synset in the hypernym chain the script will find the synsets hypernym,
and then find all the hyponyms of the hypernym.
These sibling senses are then put in a list along with the synset.

\subsection{Problems with multiple hypernyms}
When starting work on the best-fit algorithms it was discovered that some hypernyms lead to incorrect mappings.
The problem was noticed when mapping the synset quotation\#noun\#2,
described as "a short note recognizing a source of information or of a quoted passage",
which was mapped to the schema.org term MusicRecording.
This error was caused by the fact that the hypernym chain of quotation\#noun\#2 contains
section\#n\#1 which is a hyponym of both music\#n\#1 and writing\#n\#2, shown in figure \ref{IncorrectHypernyms}.

\fig{IncorrectHypernyms}{The start of the hypernym chain of quotation\#noun\#2}

The synset music\#n\#1 is described as "an artistic form of auditory communication incorporating instrumental or vocal tones in a structured and continuous manner".
The fact that quotation\#noun\#2 is a hyponym of music\#n\#1 violates the rules for hypernymy discussed in \ref{Hypernymy}.
Hypernyms are supposed to be generalizations of its hyponyms,
and they should be in a "is a" relationship to each other.
A native user of English would not say that a quotation is a type of music in the senses of the words are described above.
This error has been reported to the maintainers of WordNet and will be fixed in the next public release (C. Fellbaum, personal communication, May 1).

The main problem this caused was that it made it unclear if both hypernyms would maintain the meaning of a synset,
when the synset had more than one hypernym.
It's not possible to choose a strategy where one only picks one of the hypernyms since the system has no way of
knowing which of the hypernyms to choose.

The system now handles cases where a synset has multiple hypernyms by aborting the generation of the hypernym chain,
and only using the hypernyms found up to that point.
This cutoff can be seen in the hypernym subroutine in the Perl script in listing \ref{HypernymChainCode}.
This is not an optimal solution as it discards a lot of synsets that would give good mappings.
It was however chosen to avoid getting extra incorrect mappings when testing the system.
The solution chosen was also simpler to implement than check against hypernyms that were found to be wrong,
and a more sophisticated solution to choosing hypernyms was decided to be outside of the scope of the research problem.

\section{Best-fit mapping}
\label{BestFitMapping}
I will now describe the two algorithms that were developed to find mappings between the WordNet synsets and the ontologies.
Both algorithms will start by looking at the synset itself.
The system uses the direct mappings of the synset should they exist as there is no reason to believe that one can do better than a
direct mapping.
Given that there are ontologies that have not been mapped to the system uses the best-fit algorithms to find the missing mappings.

\subsection{Hypernyms first}
The hypernyms first algorithm traverses the hypernym chain by first looking at the direct hypernym of the synset for mappings.
It will then follow the hypernym chain upward as long as no mapping has been found.
This means that,
except in cases where the generation of the hypernym chain is aborted because of multiple hypernyms such as mentioned in section \ref{BuildingTheHypernymChain},
the algorithm will always map to the most general type of the corresponding ontologies instead of looking at the siblings.
%%% Include in discusion section on other traversals that could give better
\fig{HypernymChain}{{A hypernym chain with siblings}}
Looking at figure \ref{HypernymChain}, the hypernyms first algorithm would follow the links upward to the top.
If there are no mappings in the hypernym chain the algorithm will start to look at the siblings,
and them looking at them in the same order as the hypernyms.
It will start looking at the siblings of the synset and see if any of them have a mapping.
As long as there are ontologies without mappings it will then follow the hypernym chain upward again,
this time looking at the siblings of the hypernyms.
If the mapping that is found belongs to a sibling then the algorithm map to the super-type of the type that it maps to.

\subsection{Hypernym then siblings}
The hypernym first algorithm is conservative.
Since hypernymy is a "is a type of" relationship there is little chance that the resulting mappings can be mistaken.
It does however move quite quickly up the hypernym chain leading to mappings to types that are at higher level
of abstraction than the synset.
My other try at creating an algorithm tries to stay at a closer level of abstraction by looking at the siblings earlier
in the traversal.

The algorithm also starts by looking at the hypernym.
When a mapping is found from a sibling we use the super-type of the mapped type, not the type it self.
Since WordNet has a very large vocabulary,
it is reasonable to assume that a mapping from the hypernym would be closer or at least as close in meaning as a
mapping to the super-type found through a sibling.

The algorithm diverges from the hypernyms first by looking at the siblings of the synset after looking at the hypernym.
It will then follow the same procedure up the chain, first looking at each synsets hypernym,
then looking at its siblings before moving up the chain and repeating the process.
Looking again at figure \ref{HypernymChain} the algorithm will at each node first look at the node above,
and then at the nodes to the right, before moving one node up the tree.
The manner in which the algorithm chooses which sibling to map to is at the moment fairly naïve.
The algorithm will use the first match it finds,
meaning that if two or more siblings have mappings to an ontology the first one found by the algorithm will be used.
As shown in listing \ref{HypernymChainCode} the order of the siblings in the JSON is the same as the order that they
are given by querying the WordNet database.
For the system this is equal to using an unordered list since the order is unknown.
No reason could be found which lead to the assumption that there was a reason to prefer one mapping over another,
since there was not at this point any way for the artefact to know the meaning of the synset.
Further refinement is likely possible of this assumption does not hold.

\section{Exporting the document}
To export a page after adding metadata the user clicks on the export tab in the sidebar and on the export button.
This will provide the user with a link the Web site with the metadata added as shown in figure \ref{SidebarExported}.
\fig{SidebarExported}{The export tab after exporting the current page} % IMAGE OF EXPORT PANE
The exported site is parsed to put the JavaScript and iframes back in so that the Web site look the same as it did
before it was imported into \theartefact.
On the export pane the user is given the option to add a Google plus id to an authorship tag on the page.
This is treated as a separate task as it pertains not to the content of the page, but to the page itself.
It is also not a part of the schema.org ontology, but is included as it is a natural way to increase the visibility of
the Web site in search results using metadata.

\theartefact uses MongoDB\footnote{\url{http://www.mongodb.org}} as a database for storing the Web pages after the user has added metadata.
MongoDB is a NoSQL database that uses JavaScript as its database interaction language.
MongoDB uses a dynamic schema for storing data,
meaning that each document in a collection does not need to have the same structure.
This makes it suitable for projects with a rapid development style since the schema does not have to be
finalized at early stages of development.
The artefact uses mongoose\footnote{\url{https://github.com/learnboost/mongoose}} as a tool for modeling the objects.
Mongoose allows one to abstract a lot of the database logic away and simplifies connecting with the database.
When receiving a document to store the module removes all the extra comments that were added when the document was
imported, and puts the document back in its original state.
The database also stores the namespaces used in the document and the date on which the document was created.
When the document has been stored successfully the URL of the newly created Web page is returned to the user as shown in
figure \ref{SidebarExported} on page \pageref{SidebarExported}.
If there was an error during the process the user is alerted about what went wrong.
