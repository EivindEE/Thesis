% Chapter Template

\chapter{Introduction} % Main chapter title

\label{Introduction} % Change X to a consecutive number; for referencing this chapter elsewhere, use \ref{ChapterX}

\lhead{Chapter 1. \emph{Introduction}} % Change X to a consecutive number; this is for the header on each page - perhaps a shortened title

%----------------------------------------------------------------------------------------
%	SECTION 1: Background
%----------------------------------------------------------------------------------------

\section{Background}
Background for paper.

%----------------------------------------------------------------------------------------
%	SECTION 2: Motivation
%----------------------------------------------------------------------------------------

\section{Motivation}
At the moment, adding semantic markup to websites is unfeasible for most content creators on the web.
Adding proper meta data does not only mean that you need to know HTML, 
but also that you need to understand the concept of ontologies and set theory. 
In addition you need to know of the implementation of RDFa, microdata or microformats, 
as well as the content of the specific ontologies you intent to use on your site.


It is estimated that the amount of information created by humans before 2005, 
is smaller that the amount of information created after 2005.

I chose to develop using node.js because it's a new and exciting technology which would make it posible to write the entire app using one programming language for front end, server side and as the database query language.

Creating a web app allows me to work with both back end server side components, and front end work. 

%----------------------------------------------------------------------------------------
%	SECTION 3: Research Questions
%----------------------------------------------------------------------------------------

\section{Research Questions}
In this thesis I want to see if it is feasible to use WordNet synsets as an intermediary 
to create mappings from natural languages into formal ontologies.
The thesis will look at different strategies for mapping synsets to ontologies. 
The strategies will try to balance finding mappings which are at an equal, 
or similar level of abstraction with having a low level of incorrect mappings.
In the cases where there are incorrect mappings, 
we will try to analyse the results to see if we can find the cause of the mistakes to see if they are caused by the 
mapping algorithms, or by incomplete or erroneous mappings.
\subsection{Goals}
The tool should be able to:
\begin{itemize}
	\item Let users add meta data using natural language
	\item Let users disambiguate
	\item Create mappings into several existing ontologies
	\item Be flexible enough that new ontologies can be added
\end{itemize}

\subsection{Sub-questions}
Insert research sub-goals
