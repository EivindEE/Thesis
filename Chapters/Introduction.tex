% Chapter Template

\chapter{Introduction} % Main chapter title

\label{Introduction} % Change X to a consecutive number; for referencing this chapter elsewhere, use \ref{ChapterX}

\lhead{Chapter 1. \emph{Introduction}} % Change X to a consecutive number; this is for the header on each page - perhaps a shortened title

%----------------------------------------------------------------------------------------
%	SECTION 1: Background
%----------------------------------------------------------------------------------------

\section{Background}
Increasing smartphone usage.
More tasks possible to perform on smartphones.
Introduction of partially context aware systems (Google Now, Siri)
Modern smartphone OSes all follow the app-centric model, but with varying implementations.

%----------------------------------------------------------------------------------------
%	SECTION 2: Motivation
%----------------------------------------------------------------------------------------

\section{Motivation}
Modern smartphones works well for uncomplicated single-app tasks, but not for tasks requiring several sources of information or several different tools.
Want to explore the activity-centric computing model for a mobile context.
Want to explore how one can use context awareness to help us remember and do things.

%----------------------------------------------------------------------------------------
%	SECTION 3: Research Questions
%----------------------------------------------------------------------------------------

\section{Research Questions}
In which ways does the introduction of context-awareness and the implementation of activities influence the usefulness of a personal information management system?
In which ways does the introduction of action-centric computing and context awareness improve the perceived usefulness of the system for users.

\subsection{Goals}
I want to explore ways to test the concept using a series of low-fidelity prototypes, and present the final low-fidelity prototype as the resulting artefact of my research
I want to present a plausible system architecture for implementing the concept in practice

\subsection{Sub-questions}
How do you present defined activities and tasks to the user in order to facilitate manual activation of an activity and a task?
How do one define activities and facilitate activity creation in such a way that users intuitively understand what they are and what they do?
How do one define contexts in such a way that users understand what they are and what they do?
Should one present pre-defined possible contexts or let the users define their own contexts from scratch?
How to define the role and concept of tasks in an understandable manner?
How to facilitate the creation of tasks and how to present the task-creation process to users given their dynamic nature?
How to explain the connection between contexts, tasks and activities in a way that is understandable to the user?
Thus facilitating tinkering?
Are users with a low or average understanding of computers able to understand the concept, and if they are will they be able to tinker with the system in order to personalise it to their needs?
How can the system interact with the user without using the screen (in-pocket-interaction)? (e.g through voice commands, noises, buzzing, physical buttons)
How can one facilitate a complaint-process (for annoying/inconvenient notifications) and maybe a naive machine-learning process based on these complaints?
Should tasks have an importance measure?
In what way should the system present and facilitate manual activity-starting?
Should one present the tasks or should one show the activities, with some dialogue to fetch a task, both?
Should one limit the number of widgets per activity-shell? If so, what is the sweet spot?
Notification-interaction, is it possible, if so how, to move interaction out of the activity-shells for simple tasks?
Which elements of present smart phones does not translate to the context-activity-task model?
How can one facilitate task cooperation between multiple individuals?
“In-app-adjustments” – where should users go to adjust tasks and activities – how rigid/editable should they be?
