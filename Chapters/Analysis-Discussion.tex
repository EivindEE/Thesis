% Chapter Template

\chapter{Analysis and Discussion} % Main chapter title

\label{AnalysisAndDiscussion}

\lhead{Chapter \ref{AnalysisAndDiscussion}. \emph{Analysis and Discussion}}

%----------------------------------------------------------------------------------------
%	Design Research
%----------------------------------------------------------------------------------------
\section{Comparing the algorithms}
\label{ComparingAlgorithms}
To compare the two algorithms we generated a list of english nouns
which we sent through the lexitags server to get synsets that corresponded to the meanings of each word.
Both these lists were preprocessed to remove duplicates and to format them as javascript objects
\footnote{\url{https://github.com/EivindEE/Madame/tree/master/testing}}.
The final list of synsets contained 4350 unique synsets.
We wrote a short script that we used to run the synsets through the best fit algorithms,
and write a report of the results.
For the schema.org version of the test we wrote the average depth of the mapped type,
as well as the total number of times the two algorithms had the same and different mappings.
The debth was calculated as the distance from the root node in the tree,
i.e. schema:Thing it self had a depth of 0,
schema:Person which inherits directly from schema:Thing has a depth of 1 and so on.
For the SUMO version the depth of each type was unavailable,
so we only have numbers showing the agreement between the algorithms.
The full results from the tests can be found at the URL \url{https://github.com/EivindEE/Master-thesis/tree/master/AlgComparison}.

The results from the SUMO test display no difference between the two algorithms when mapping from WordNet to SUMO.
This indicates that there must have beena mapping either directly from the synset,
or from the direct hypernym of the synset for each of the 4350 synsets in our test.

The schema.org results are more interesting.
Our predictions beforhand was that the hypernymes first approach would have fewer incorrect mappings,
but would give results at a more shallow depth.
The last prediction was the easiest to test.
For each mapping we registered the depth of the type in the schema.org hierarchy.
These depths were averaged over the total number of mappings made.

As seen in table \ref{table:AlgorithmComparison} both algorithms map fairly high in the hierachy.
The hypernymes first approach maps to a type at level 0.69 on average when considering all the synsets,
or to a type at level 0.72 when ignoring the cases where the two algorithms gave the same result.
As predicted the hypernym then siblings approach does a little better, though not much,
mapping to types at level 0.8, or 1.2 when excluding identical mappings.
Looking at the data it is obvious that the hyponymes first approach much more frequently leads to mappings to Thing and Intangible.
The hypernym first algorithm maps to Thing 449, and Intagible 481 times,
while hypernymes then siblings maps to Thing 334 and Intangible 81 times.
Neither Thing nor Intangible are very interesting mappings in the ontology.
As described in \ref{schemadotorg}, Thing is the most general category, meaning that every concept belongs to this category.
Intangible is described\footnote{\url{http://schema.org/Intangible}} as "a utility class that serves as the umbrella for a number of 'intangible' things",
and does not have any special properties in the ontology.
Even if these mappings are quite uninteresting it is not always obvious that there are types in schema.org which are better fits for the synsets.


\begin{table}[h]
	\centering
	\begin{tabular}{lll}
										& Hypernyms First 	& Hypernym then siblings	\\
		\emph{For all}					&					&							\\
		Avg. depth total				& 0.688506 			& 0.804138					\\
		No mapping found				& 598				& 598						\\
		\emph{For different mappings}	&					&							\\
		Avg. depth different			& 0.721507			& 1.183823					\\
		Mappings to "Thing"				& 449				& 334						\\
		Mappings to "Intangible"		& 481				& 81						\\
		Errors (first 250 mappings)		& 2 				& 114 						\\
		Error rate (first 250 mappings)	& 0.8\%				& 45.6\%					\\
	\end{tabular}
	\caption{Comparison of the mapping algorithms}
	\label{table:AlgorithmComparison}
\end{table}

Noun list from http://www.desiquintans.com/articles.php?page=nounlist

https://github.com/EivindEE/Master-thesis/blob/master/AlgComparison/AlgComparisonResultsBig
https://github.com/EivindEE/Master-thesis/blob/master/AlgComparison/compare-different
%----------------------------------------------------------------------------------------
%	Design Research
%----------------------------------------------------------------------------------------
\section{Suitability of WordNet for mapping natural language}
