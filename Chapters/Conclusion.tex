% Chapter Template

\chapter{Conclusion} % Main chapter title
\label{Conclusion} % Change X to a consecutive number; for referencing this chapter elsewhere, use \ref{ChapterX}
\lhead{Chapter \ref{Conclusion}. \emph{Conclusion}} % Change X to a consecutive number; this is for the header on each page - perhaps a shortened title
We are now going to summarize our

\section{Findings}

We have found that using siblings of synsets as a basis for mapping to ontologies is to naïve.
Our study has found no reason to believe that siblings reliably tell us anything about the semantic properties of a synset.

Using hypernyms as a basis for mapping gives correct mappings.
The approach does however result inn high level mappings, and could benefit from further refinement.

We found that WordNet was unable to capture the gramatical number of natural language.
This means that the tool won't have any way to distinguish between ontological types which differ in this respect.

The metadata we created using \theartefact\ was comparable to that pressent at current websites which use schema.org to
enrich their content.

We found that we could add metadata to the webpages without changing the way they were displayed in the browser.

\section{Reflections}

\section{Further work}
The goal of this thesis has been to create a functioning prototype of an artefact that allows users to add metadata
to webpages by using natural language.
As argued we mean that we have managed to create a system that fullfils the goals we set at the outset.
we now see several new interesting ways the tool could be developed further to increase its value to users.

It would be interesting have mappings to more ontologies,
and one could offer the user a chance to say what the topic of the page was.
In this way one could offer mappings to the Friend Of A Friend ontology if it was a webpage dealing with
social interaction, the Good Relations ontology if it was a commerce page and so on.
To do this the mapping module should get further development to complete the process of uncoupling the ontologies
from the code.

We should develop a way to allow for multiples of a single property on the webpage.
The idea of adding properties came quite late in the project,
and is as a consequence not as feature rich as it should be.
One should also research into finding some way of allowing for properties from other ontologies.
One difficulty here would be finding a way of presenting these without overwhelming the user.

Adding multiples might also mitigate the issue that synsets aren't regarded as distinct because of their gramatical number.
For schema.org the issue of aggregated terms is limited as it only has two types which are the aggregation of multiples of a type.
By allowing multiples of properties, we could handle adding the aggregation of these behind to the document automatically.
This would be a good solution for the issue with schema.org, but it might not scale well if we include other ontologies
that separate concepts by way of gramatical number.

When the website has experienced more usage it would be exciting to examine the usage logs to see which types of text gets tagged.
Actual usage data would be an interesting source to discover concepts that users frequently want to map.
Examination of these logs could therefor be a useful starting point to find out which ontologies to create mappings for,
and which parts of these ontologies which would create the most value for the users.
The usage data we have allows us to extrapolate text which we did not find good disambiguations for
by assuming that this is text that was selected but where the user didn't click any of the suggested senses.
We can also count the frequencies at which different synsets or DBPedia terms were chosen as the concept the
user wanted to describe, and use this to target the work of mapping.


